\section{Introduction}
\label{sec:introduction}

The cloud is changing how users store and access data.  This is true
for both legacy applications that are migrating local data 
into the cloud, and for emerging cloud-hosted applications
that use a combination of content distribution networks (CDNs) and
client-side local storage to achieve scalability.  In both cases, the goal is to 
compose existing storage and caching services to
implement scalable storage for the application.  However, the
challenges in doing so are to keep data
consistent, keep the storage layer secure, and
enforce storage policies across services.  This paper 
presents \Syndicate, a wide-area storage system 
that meets these challenges in a general, coherent manner.

There are two reasons to compose existing services to implement scalable storage. 
First, if we factor a scalable storage system by the
capabilities of existing services,
it is interesting to ask what extra functionality
is needed, and how is it best provided?
Our strategy is to use cloud storage for durability and 
scalable capacity, edge caches (CDNs and caching proxies)
for scalable read bandwidth and reduced upstream load,
and local storage for fast, possibly offline access.
Once these capabilities are met, the remaining
desirable functionality is wide-area consistency,
storage layer security, and storage policy enforcement.
\Syndicate\ introduces a new Metadata Service and storage
middleware that provides this functionality in a general,
configurable way.

The second reason is that leveraging existing services
is more than just taking advantage of someone having 
already implemented the corresponding functionality.
More importantly, doing so also leverages a globally-deployed
and professionally-operated instantiation of that functionality.
Deploying and operating a global service, even with 
the availability of geographically distributed VMs, is an
onerous task.  \Syndicate\ side-steps that problem by 
using existing storage and caches, and in the process,
significantly lowers the barrier-to-entry to constructing a 
customized, global storage service.

The thorniest problem \Syndicate\ faces is contending 
with the weak consistency offered by edge caches.  Namely,
the cache operator, not the application, ultimately decides
what constitutes ``stale'' data for eviction purposes, making
it difficult to rely on cache control directives.
Rather than trying to avoid caches, we intentionally incorporate 
them into the solution due to the tangible benefits they 
offer.  Already, enterprises deploy on-site caching proxies to 
accelerate Web access for users, content providers employ 
global CDNs to distribute content to millions of readers,
and ISPs deploy CDNs in their access networks to avoid 
transferring frequently-requested data over expensive links
~\cite{akamai,coralcdn,coblitz}.

Our key insight into composing existing storage and caching
services is that we should {\it avoid} treating them as
first-class components.  Instead, we must treat 
them as interchangeable utilities, distinguishable
only by how well they provide their unique capabilities.
Then, applications select services arbitrarily,
and layer consistency, security, 
and policies ``on top'' of them with \Syndicate.
By doing so, we provide them a virtual cloud storage
service.

Our key contribution is a novel consistency protocol, employed in both the data
plane and control plane, that achieves this end.  With this protocol,
\Syndicate\ overcomes the weak consistency of edge caches in the data plane
while continuing to leverage the benefits they offer.  It also 
lets \Syndicate\ leverage caches in the control plane to scalably distribute
certificates and signed executable code to middleware end-points,
in order to secure the data plane and enforce data storage policies in 
a trusted manner.
