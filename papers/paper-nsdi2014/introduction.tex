\section{Introduction}
\label{sec:introduction}

The cloud is changing how users interact with data.  This is true for legacy Web applications that are migrating on-site data sets to cloud storage, as well as for emerging Web applications that are augmenting cloud-hosted storage with edge caches.  In both cases, users stand to benefit by leveraging multiple cloud storage and edge caching providers to harness already-deployed infrastructure, thereby instantly gaining a global footprint.  Doing so, however, introduces several design challenges related to storage abstraction, data consistency, data durability, and data access control.  This paper describes \Syndicate, a wide-area storage system that addresses these challenges in a coherent manner on behalf of applications.

The functional benefits offered by cloud storage and edge caches are well understood.  Cloud storage offers users an ``always-on'' repository for hosting a scalable amount of data, while keeping it consistent with writes and limiting access to sensitive data.  Edge caches distribute user data from origin servers (including a cloud storage provider's datacenters) to a scalable number of readers at the edge of the Internet, offering higher aggregate bandwidth than what the origin servers alone could provide.

In addition to functionality, cloud storage and edge caching providers offer key storage utility benefits to applications.  Each cloud storage provider lets an application host data at several geographically distributed sites on generously-provisioned datacenters (improving data durability), and each edge caching provider lets an application cache copies of its data in many PoPs around the world and serve them to nearby users (improving data locality).

Because of these utility benefits, application developers stand to gain from leveraging multiple providers at once to meet their storage needs.  This lets them maximize
utility beyond what a single one provides, and lets them add and drop providers at will to find the optimum cost/utility trade-off.  Moreover, upgrades to a provider automatically benefit the application.

However, leveraging multiple providers introduces challenges to reaping their individual functional benefits.  An application can no longer rely on a single provider to ensure 
data consistency, durability, and integrity, and must enforce data access controls separately.  Additionally, the application must carefully distribute data across providers so it 
can scale reads and writes while tolerating provider unavailability.
This is all in addition of implementing the domain-specific storage requirements (e.g. logging,
encryption, formatting, and so on) that not all providers offer.

Our key insight is that rather than treating a cloud storage provider as a
first-class component (i.e. a storage layer, a storage peer, or a virtual disk),
we can instead treat a set of cloud storage providers
as an underlying medium that is only concerned with transparently enhancing data durability.
Similarly, we can treat a set of caching providers as an underlying medium that is
only concerned with transparently enhancing data locality.  
Then, applications gain both the functional and utility benefits of the cloud by
composing providers for utility benefits, and deploying the desired functionality 
as a first-class but provider-agnostic storage layer.  To achieve this, we created \Syndicate.

The key contributions of \Syndicate\ are a wide-area storage architecture and data consistency protocol that runs on top of underlying caching and storage providers.  \Syndicate\ provides an interface for accessing and governing the usage of a scalable amount of data across providers while retaining application-specific storage functionality.  The consistency protocol organizes data into sets of directory hierarchies and lets users choose the trade-off between data consistency and read performance independent of underlying caches.

This paper first explores the case for composing multiple cloud storage and edge cache providers into a single coherent storage system through a set of use-cases (Section~\ref{sec:motivation}).  From the use-cases, it argues for a set of abstractions for interacting with cloud storage and edge caching providers, and presents an architecture and data consistency protocol that work with them (Section~\ref{sec:design}).  It then presents the implementation and evaluation of a prototype system (Section~\ref{sec:implemenation} and Section~\ref{sec:evaluation}), and shows how \Syndicate\ differs from related work (Section~\ref{sec:related}).  It concludes in Section~\ref{sec:conclusion}.



%The cloud is changing how users store and access data. This is true for legacy applications that are migrating on-site data sets to cloud storage to increase their durability and availability, as well as for emerging applications that are augmenting cloud-hosted storage with edge caches to reduce client latency and scale aggregate bandwidth. In both cases, the challenge is to let users leverage a combination of cloud-hosted storage, local on-site storage, and distributed network caches to address storage consistency, durability, availability, scalability, and performance requirements. This paper describes \Syndicate, a wide-area storage service that composes existing storage and caching mechanisms to address these challenges in a coherent manner.

%The value of network caches and CDNs to read-heavy wide-area workloads is well understood. They are used to reduce read latency, increase read bandwidth (both local and aggregate), and lower the cost of transferring frequently-requested data over expensive links. These gains are great enough today that many enterprises deploy caching proxies for their on-site users, and content providers leverage global CDNs to scale content delivery to their customers~\cite{Akamai, coralcdn, coblitz}.

%The downside of contemporary network caches is that they weaken data consistency by continuing to serve stale data after a write completes. Moreover, each cache/CDN operator has a separate policy on what constitutes ``stale'' data for eviction purposes, which neither readers nor writers control.\footnote{The cost of honoring application-given cache directives is significant enough in practice that at least one popular HTTP cache implementation~\cite{Squid} allows its administrator to selectively ignore them, in violation of the HTTP specification~\cite{HTTP-RFC}, just to avoid the excessive performance loss that honoring them could cost.} This makes them difficult to use in wide-area read/write settings, even when there are many readers. This is because multiple cooperating origin servers (e.g. users, application processes, etc.) require stronger consistency than what the intermediate caches offer.

%To address this problem, we created \Syndicate, a storage service that lets users compose local storage, cloud storage, and network caches to build a coherent wide-area storage system. \Syndicate\ is designed to run on top of user-chosen infrastructure and expose user data via a common interface, while offering good performance in read-heavy but read/write wide-area workloads.

%The key contribution of \Syndicate\ is a wide-area data consistency protocol that works independently of underlying caching and storage mechanisms. The protocol lets users choose the trade-off between data consistency and read performance, as well as between write durability and write performance. It ensures that writes are observed in the same order by user processes while letting users hit unmodified data in network caches, and it ensures readers observe data consistent with all writes after a user-specified amount of time. It additionally offers a primitive operation to let readers and writers implement cooperative mutual exclusion for application-specific consistency semantics.

%The paper first explores the problem space through a set of use cases and describes a system architecture that meets the requirements of these examples (Section~\ref{sec:motivation} and~\ref{sec:design}), and then outlines and evaluates a specific implemenation that adheres to this design (Section~\ref{sec:implementation} and~\ref{sec:evaluation}). We call attention to these two perspectives---design focus and implementation focus---because they correspond to two different ways to think about \Syndicate. On the one hand, \Syndicate\ is designed to be a general-purpose storage system that leverages existing mechanisms. On the other hand, when configured with a POSIX file system interface, \Syndicate\ effectively implements a wide-area file system, with the specific semantics implied by POSIX. Both perspectives are accurate.

