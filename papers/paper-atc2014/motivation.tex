\section{Motivation}

Any secure communication protocol needs to provide three 
fundamental security guarantees of message confidentiality 
(only sender and receiver can read the message), message 
authenticity (receiver can verify that the message came 
from the correct sender), and message integrity (the message 
was not tampered with during communication). We call these 
CIA guarantees. Traditional email protocols like SMTP do not
provide them.

CIA guarantees can be achieved by using an out-of-band security 
protocol like PGP on top of email. However, this significantly 
complicates the user experience and is not a practical approach 
for a majority of email users. Most email users are not tech-savvy 
and over the years they’ve become accustomed to the ease-of-use of 
web-based email services like Gmail~\cite{GMail} and Hotmail~\cite{hotmail}. Providing CIA 
guarantees while keeping the system as easy to use as webmail is 
a non-trivial technical problem. 

In this paper, we focus on providing end-to-end CIA guarantees 
with a user experience as close as possible to that of webmail. 
Specifically, we focus on providing CIA guarantees on the message 
contents (subject, body, and attachments) and the account state only. 
Providing security guarantees for metadata, such as when the message 
was sent and received and what the senders’ and recipients’ addresses 
are, is the subject of future work.

We first describe our threat model and present possible adversaries 
that two webmail users Alice and Bob are likely to face. From the 
threat model, we derive the requirements on key management. We also 
define usability requirements of Alice and Bob for various email tasks.
 Our goal is that both the security requirements and the usability 
requirements must be met. We then argue that to perform the required 
key management automatically and securely, a webmail service must 
have at least three separate non-colluding services---one for processing 
state, one for hosting state, and one for hosting copies of the public keys. 

\subsection{Threat Model}
Let Alice and Bob be two users that rely on webmail to communicate.  They do not understand key management, but they never divulge their passwords. They each use multiple different user devices (endpoints) to access their accounts, and have fixed expectations on how to carry out common tasks in webmail (Table~\ref{tab:user-experience}).

Let Eve and Mal be technically capable adversaries of Alice and Bob.  Eve wants to break message confidentiality by reading as many messages as possible.  Mal wants to break integrity and authenticity by altering or forging messages.  However, they are computationally bound, they cannot compel users to divulge information, and they want to avoid getting detected.

We assume that it is more cost-effective for Eve and Mal to target email servers and communication channels than (potentially millions of) user devices i.e., attempting to compromise an email server that hosts information for $n$ users is more cost effective than trying to compromise those $n$ users individually. Also, from a practical point of view, most messages do not have much information of interest to Eve and Mal.

Eve’s main method of attack is eavesdropping.  With one exception, she can read traffic on every network link and server, as well as every bit of information exchanged between users without being detected.  She can compel any server or network link to disclose all of its information to her over any length of time, and can remember an arbitrary amount of data.  The only bit of information she cannot observe is Alice's password when she registers her account.

Additionally, we assume that Eve can make offline copies of any endpoint’s stable storage, but she cannot observe the state of the user devices while they are in use. In other words, we assume that Eve doesn’t have access to decrypted private keys.  

Unless otherwise noted, we assume that at any given time, Mal can either alter the behavior of servers, or alter packet flows in communication channels, but not both at once. We assume that if Mal compromises servers, she cannot compromise at least one of the STEAK servers Alice and Bob rely on. Similarly, if Mal compromises communication channels, she cannot alter the packets in at least one of the channels Alice and Bob use.  Alice and Bob do not know which servers or channels are uncompromised.  Further, Mal cannot cause endpoint devices to misbehave while they are in use.

We will show in subsequent sections that under this threat model, STEAK provides Alice and Bob message and account state CIA.  Eve cannot feasibly learn the contents of their conversations, and Mal can only prevent the users from communicating.

\subsection{Key Management and Usability}
Even though we only care about CIA guarantees on messages and account 
state, the key management necessary to provide them, given our threat 
model, affects other necessary tasks like session management, account 
management, and contact management in addition to basic sending and 
receiving of email (Table~\ref{tab:user-experience}). However, Alice
expects to use any endpoint to access
their email, endpoints cannot preserve hard state across sessions.  
Thus, Alice’s endpoint must securely obtain not only Bob’s public key, 
but also her private key each time she logs in.

\begin{table}[ht!]
\begin{tabular}{ | l | L{4.5cm} |}
\hline
\textbf{Task} & \textbf{Key activity} \\
\hline
Log in & Unseal private keys \\
Log out & Clear private keys \\
\hline
Create account & Create private keys, revocation certs; publish pubkeys \\
Delete account & Publish revocation cert \\
Change password & Reseal private keys \\
Reset password & Regenerate private keys, or recover with security questions \\
\hline
Add contact & Obtain and verify pubkey; watch for revocation \\
\hline
Write email & Encrypt and sign \\
Read email & Verify and decrypt \\
\hline
\end{tabular}
\caption{\it Common webmail tasks and the requisite key management to perform to gain CIA.}
\label{tab:user-experience}
\end{table}

At the same time, Alice and Bob do not know how to use or manage keys. 
They expect to submit their usernames and passwords to begin a session 
in their web browsers, and gain CIA guarantees automatically.  They 
also expect to use any user device and always see up-to-date information.  
This means the user endpoints must synchronize hard state across sessions, 
including the keys, account information, and messages, with one or more 
common always-on data repositories.  Moreover, they must do so in a 
tamper-evident way to prevent Mal from silently altering state, and 
in a privacy-preserving way to prevent Eve from reading it.

If Alice and Bob communicate through a single server, Mal can trick 
their endpoints into learning the wrong public keys if she compromises 
the server or one of the links.  But under our threat model, Alice 
and Bob can only use one server if they first assume that Mal does 
not compromise anything!  Thus, exchanging public keys must involve 
leveraging more servers and links than Mal can compromise, and in 
practice this can be assumed to be at least 1.  This means that 
there must be at least two sources for both public keys and and 
anti-tampering metadata.

Because Eve can read any server and any network link, the endpoints 
must necessarily implement end-to-end state encryption between Alice 
and Bob.  This prevents the (computationally bound) servers from 
processing messages, however, so the endpoints must work together 
to do so instead, constituting a separate message processing subservice 
within the email system.  Thus we need at least three components in the 
design of our system: a trusted message processing component at the 
endpoints, and at least two public key and anti-tampering metadata 
repositories.

Using our strategy, we reduce the problem of giving Bob Alice’s public 
key to helping Bob’s endpoint discover which repositories can serve it.  
We address this by embedding the names of the repositories directly into 
Alice’s email address. We realize that embedding this extra information 
into the email address makes the email address harder to read/remember 
and we plan to address this issue in future work. 
