\section{Evaluation}

Our usability goal has been to preserve the semantics of webmail while offerring transparent end-to-end CIA.  To evaluate this, we consider the steps and information required to complete common webmail tasks, and compare them to how STEAK and PGP perform them.  These include setting up and destroying an account, adding and removing a contact, sending and receiving messages, and changing account passwords.

In evaluating against PGP for usability, we use the user’s guide for PGP 7.0 as a baseline comparison.  We also consider approaches taken by Enigmail, Mailpile, and Mailvelope, which seek to make PGP easier to use for securing email.  We show that while STEAK does not offer exact webmail semantics, it offers semantics much closer to webmail than even user-friendly PGP implementations.

At the same time, our performance goal for STEAK is to avoid hindering responsiveness to user interaction.  Our preliminary evaluation shows that running AutoKey asynchronously hides the time needed to perform key distribution, and that encryption and decryption do not significantly increase latency.

\subsection{Endpoint Management}
Assuming the web browser is already installed, setting up webmail takes no effort from the user.  However, because performing encryption in server-chosen Javascript with no validation is unsafe, both PGP and STEAK require an additional software component to be installed on each endpoint.  While this can be simplified with the endpoint device's app store, a PGP user must also create a subkey for each device, and sign it with the master key.

\begin{table}[ht!]
\begin{tabular}{ | L{2.0cm} | L{4.5cm} |}
\hline
\textbf{Webmail} & \textbf{STEAK and PGP} \\
\hline
(None) &

\vspace{-3mm} 
\begin{enumerate}
  \item{Install endpoint software.}
  \item{(PGP) generate and sign subkey.}
\end{enumerate} \\

\hline
\end{tabular}
\caption{\it Steps to set up an endpoint.}
\label{tab:account-creation}
\end{table}

Destroying an account in webmail and STEAK are equivalent--the user navigates to the page in the UI to do so, and enters the username and password.  The rest is automated.  In PGP, the email account must be destroyed separately from the key.  The private keys must also be erased, and the public keys revoked.

\subsection{Account Management}
Registering an account in STEAK has a similar workflow to registering a webmail account; the only difference is that STEAK users must also submit cloud storage authentication tokens and additional metadata repositories (if desired).

By contrast, a PGP client requires the user to create an email account before configuring it, making the task at least as difficult as webmail.  It additionally requires users to generate keys and revocation certificates, and export the public key (e.g. to a key server).  While it is possible to integrate this into the account registration process (the approach taken by Mailpile), the user is inevitably involved in the key setup process because she will need to learn how to sign and trust other users’ keys. 

\begin{table}[ht!]
\begin{tabular}{ | L{3.5cm} | L{3.5cm} |}
\hline
\textbf{Webmail and STEAK} & \textbf{PGP} \\
\hline
\vspace{-3mm}
\begin{enumerate}
  \item{Navigate to provider URL.}
  \item{Submit username, password, out-of-band channel (STEAK: also cloud storage creds and metadata repositories).} 
  \item{Activate account with information sent on out-of-band channel.}
\end{enumerate} 
\vspace{-\topsep}&

\vspace{-3mm}
\begin{enumerate}
  \item{Create email account (steps 1-3 in webmail).}
  \item{Configure PGP client to use the email account.}
  \item{Generate and encrypt keys.}
  \item{Generate revocation cert.}
  \item{Publish pubkey.}
\end{enumerate} 
\vspace{-\topsep}\\

\hline
\end{tabular}
\caption{\it Steps to create an account}
\label{tab:account-creation}
\end{table}


Changing or resetting a password in STEAK is a matter of either submitting the old password or answering the security questions, and waiting for the account state to be re-sealed.  This is similar to webmail’s semantics.  While some PGP implementations offer a key recovery option using security questions, it must be manually enabled.  Moreover, the user must manually change the key password after recovering the key.


\begin{table}[ht!]
\begin{tabular}{ | L{3.5cm} | L{3.5cm} |}
\hline
\textbf{Webmail and STEAK} & \textbf{PGP} \\
\hline
\vspace{-3mm}
\begin{enumerate}
  \item{Go to to reset form.}
  \item{Answer security questions.} 
  \item{Obtain reset URL out-of-band.}
  \item{Enter new password.}
  \item{(STEAK) Wait for re-sealing.}
\end{enumerate} 
\vspace{-\topsep} &

\vspace{-3mm}
\begin{enumerate}
  \item{Enable key recovery option.}
  \item{Contact recovery server.}
  \item{Answer security questions.}
  \item{Change key password.}
\end{enumerate} 
\vspace{-\topsep} \\

\hline
\end{tabular}
\caption{\it Steps to reset a password.}
\label{tab:account-creation}
\end{table}

Deleting an account in webmail is as simple as entering one’s password and confirming the request.  This is also the case in STEAK, where all public keys are automatically revoked once the request is confirmed.  In PGP, however, the user must not only delete the email account, but also any private keys associated with it.  The user must erase the public keys from the key servers, and publish revocation certificates for the private keys.

\subsection{Contact Management}

Adding a contact in webmail is usually a matter of filling out a form, if it is not done automatically.  At a minimum it requires an email address and a name, but other fields are possible.

With PGP, the user must obtain, verify and trust a public key as well.  Mailpile attempts to reduce this to scanning QR codes or obtaining them from email headers, but nevertheless the user must make a choice on the trustworthiness of the peers who signed it.  Applying TOFU semantics makes it easier to use but less secure than STEAK, because there is only a single source for the key.  Moroever, PGP users must periodically re-acquire keys when old ones expire.

AutoKey lets STEAK obtain keys automatically and keep them fresh.

\begin{table}[ht!]
\begin{tabular}{ | L{3.5cm} | L{3.5cm} |}
\hline
\textbf{Webmail and STEAK} & \textbf{PGP} \\
\hline
\vspace{-3mm}
\begin{enumerate}
  \item{Open new contact.}
  \item{Add name, email.} 
  \item{Save.}
\end{enumerate} 
\vspace{-\topsep} &

\vspace{-3mm}
\begin{enumerate}
  \item{Open new contact.}
  \item{Add name, email.}
  \item{Obtain, verify, and trust pubkey.}
  \item{Save.}
  \item{Repeat 3-5.}
\end{enumerate} 
\vspace{-\topsep} \\

\hline
\end{tabular}
\caption{\it Steps to add a contact.}
\label{tab:account-creation}
\end{table}

\subsection{Accessing, Sending, and Receiving Email}

Accessing email from multiple devices is trivial in webmail if the device has a web browser.  It is equivalently easy in STEAK and PGP once the endpoint code is installed and set up.

Sending a message in webmail is a matter of opening a window to compose it, selecting the recipients, and sending it.  However, PGP implementations require the user to at least click through UI elements to encrypt and sign the message.  This is because only the user knows whether or not the recipient will be capable of decrypting the ciphertext.  STEAK's transparent backwards compatibility lets users avoid this, akin to webmail.

\begin{table}[ht!]
\begin{tabular}{ | L{3.5cm} | L{3.5cm} |}
\hline
\textbf{Webmail and STEAK} & \textbf{PGP} \\
\hline
\vspace{-3mm}
\begin{enumerate}
  \item{Open compose window.}
  \item{Select recipients.} 
  \item{Type body, add attachments.}
  \item{Send message.}
\end{enumerate} 
\vspace{-\topsep} &

\vspace{-3mm}
\begin{enumerate}
  \item{Open compose window.}
  \item{Select recipients.}
  \item{Type body, add attachments.}
  \item{Encrypt and sign.}
  \item{Send message.}
\end{enumerate} 
\vspace{-\topsep} \\

\hline
\end{tabular}
\caption{\it Sending a message.}
\label{tab:account-creation}
\end{table}


Reading a message in webmail is a matter of selecting the unread message.  This is also the case in STEAK-to-STEAK communication, and STEAK/SMTP communication where no CIA is needed.  Even in STEAK/SMTP communication with limited CIA, submitting a shared password to access a message is not a foreign concept.

PGP implementations can decrypt messages automatically if the key is known.  However, the user may be prompted to obtain and trust the key if not.

\begin{table}[ht!]
\begin{tabular}{ | L{3.5cm} | L{3.5cm} |}
\hline
\textbf{Webmail and STEAK} & \textbf{PGP} \\
\hline
\vspace{-3mm}
\begin{enumerate}
  \item{Select message.}
  \item{(STEAK/SMTP): submit CIA secret.}
\end{enumerate}
\vspace{-3mm}
 &

\vspace{-3mm}
\begin{enumerate}
  \item{Select message.}
  \item{If not trusted, obtain, verify, and trust pubkey.}
\end{enumerate} 
\vspace{-\topsep} \\

\hline
\end{tabular}
\caption{\it Reading a message.}
\label{tab:account-creation}
\end{table}

\subsection{Performance}

Keeping latency as low as possible is important for the user experience.  Our strategy for doing so is to perform as much work asynchronously as possible, so as to not keep the user waiting.  However, encrypting messages must occur synchronously, because the user expects messages to be sent as soon as the "send" command is given.

