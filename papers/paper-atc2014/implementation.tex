\section{Implementation}
\label{sec:implementation}

The individual technologies needed to implement STEAK already exist.  For storage, STEAK relies on Syndicate~\cite{syndicate} to provide an abstraction layer over storage providers while implementing common consistency, integrity, authenticity, and authorization semantics.  For running the endpoint code, STEAK relies on now-ubiquitous virtual machine technology to achieve portability and isolation from the browser and underlying OS.

The two usability challenges to overcome are setting up the endpoint VM and getting access to the user’s cloud storage.  To address the former, we deliver endpoint code as an “app” that the user installs via a device-specific app store.  Once installed, it does not have to be accessed directly, and the app store will keep it up-to-date.  We believe users are already used to installing software from app stores, and thus we do not believe this extra step poses a significant usability barrier.

Our endpoint’s runtime requirements are flexible.  While a full-blown virtual machine is sufficient, the endpoint can also run in an OS container, a user-mode Linux instance, or a Portable Native Client (PNaCl)~\cite{pnacl} browser plugin.  Our prototype supports all but the last option, which is under development.

We leverage Syndicate to make it easy to access cloud storage, and to simplify our storage layer’s implementation.  To set up STEAK, the endpoint and Syndicate execute a protocol similar to OpenID whereby Alice authorizes STEAK to create and access a Syndicate volume.  The metadata repository implementation employs a similar strategy.

Regarding deployments, a STEAK server is meant to serve an organization.  Because it can read sender email addresses and source IP addresses, it employs techniques similar to those used in SMTP to rate-limit and black/whitelist external malicious users.  For SMRP clients, the server additionally verifies the message metadata record signature before accepting the request, and blocks senders who do not have known public keys.  The server deals with malicious traffic on the SMTP gateway using conventional techniques.

Regarding spam, since Bob can be expected to check his account frequently (once per day), it is unlikely that his message metadata records on his STEAK server will consume too much space.  If space becomes a concern, the STEAK server can opt to remember only the unique set of senders, in which case Bob will scan their cloud storage accounts for the messages at the cost of latency.

To enhance spam filtering, we are investigating the feasibility of privacy-preserving data mining, to allow many users to train a shared spam classifier.  One possible solution is a distributed privacy-preserving Na\"{i}ve Bayes classifier~\cite{privacy-preserving-naive-bayes}.

Key plaque is a known problem with key servers.  Metadata repositories prevent it by periodically querying the issuer's STEAK server for the key, and automatically erasing certificates with built-in expiration dates.

\subsection{Prototype}
Our prototype is in an early alpha state, but is under active development.  The endpoint and server are implemented as daemons in Python, with 6,000 and 1,000 lines of code respectively.  The UI is implemented with Google Web Toolkit, and contains about 2,000 lines of Java. We use the PyCrypto package for cryptography, and rely on Syndicate’s Metadata Service as a metadata repository.  We are in the process of integrating the logic for handling attachments, mailing lists, searching, and filtering.

The STEAK server and metadata repository will be compatible with multiple PaaS providers, to allow them to horizontally scale to support large organizations.  Users will have the option to write public keys and certificates to one or more cryptocurrency blockchains, greatly increasing the number of servers Mal must compromise to change it.
