\section{Implementation}

The individual technologies needed to implement STEAK already exist.  For storage, STEAK relies on Syndicate to provide an abstraction layer across arbitrary cloud storage providers while providing common consistency, integrity, authenticity, and authorization semantics.  For running the endpoint code, STEAK relies on now-ubiquitous virtual machine technology to achieve portability and isolation from the browser and underlying OS.

The two usability challenges the implementation must address are in setting up the endpoint VM and getting access to the user’s cloud storage and metadata repositories.  To address the former, we deliver endpoint code as an “app” that the user installs via a device-specific app store.  Once installed, it does not have to be accessed directly, and the app store will keep it up-to-date.  We believe users are already used to installing software from app stores, and thus we do not believe this extra step poses a significant usability barrier.

Our endpoint’s runtime requirements are flexible.  While a full-blown virtual machine is sufficient, the endpoint can also run in an OS container, a user-mode Linux instance, or a Portable Native Client (PNaCl) browser plugin.  Our prototype supports all but the last option, which is under development.

We leverage Syndicate to make it easy to access cloud storage, and to simplify our storage layer’s implementation..  Alice simply authorizes STEAK to create a Syndicate volume for hosting its data.  To do so, STEAK and Syndicate execute a protocol similar to OpenID whereby Alice authenticates to Syndicate on behalf of STEAK in order to tell it to carry out the storage setup.  Our metadata repository implementation uses a similar strategy for authorizing STEAK to communicate with it for the first time.

Regarding deployments, a STEAK server is meant to serve an organization.  Because it can read sender email addresses and source IP addresses, it employs techniques similar to those used in SMTP to rate-limit and black/whitelist external malicious users.  In SMRP, the server additionally has the power to verify the message metadata record signature before accepting the request, and block senders who do not have known public keys.  The server deals with spam traffic on the SMTP gateway using conventional techniques.

Key plaque is a known problem in key servers.  Metadata repositories prevent this by periodically querying the issuing STEAK server for the key, and automatically erasing certificates with a built-in expiration dates.

\subsection{Prototype}
The STEAK endpoint and server prototypes are implemented as daemons in Python.  The former contains about 6,000 lines of executable code, and the latter contains about 1000.  The UI prototype was developed with the Google Web Toolkit, and contains about 2,000 lines of executable Java. The Syndicate cloud storage system that STEAK leverages contains about 35,000 lines of C++ and 25,000 lines of Python.  Our prototype relies on Syndicate’s Metadata Service to provide the same functionality as the metadata repository.

The STEAK server and additional metadata repository implementations will eventually be compatible with multiple PaaS providers, allowing them to horizontally scale to support large organizations.  They will also be able to back up public keys and certificates to one or more cryptocurrency blockchains, under the assumption that the blockchain is infeasible to compromise even for state actors.
