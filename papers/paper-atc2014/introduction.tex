\section{Introduction}

Over the past decades, email has gained wide-spread adoption 
and is now a core communication mechanism for society. Millions of people rely on email as the primary means of communication for their jobs and billions of email messages are sent everyday. However, 
some of the original design choices of email have had a profound impact 
on email security and storage---making email a fundamentally insecure communication channel. In this paper, we revisit these design choices and propose a backwards-compatible email system, 
called StealthMail, that addresses security limitations in the context of 
contemporary usability expectations.

Users send a wide variety of information in their email messages, 
including confidential and personal information and legally-binding 
documents. Traditional email protocols, like SMTP and IMAP, do not 
provide adequate security guarantees and average email users are 
sometimes not even aware that they're communicating over a fundamentally 
insecure channel. In a secure communication channel, when Alice sends 
Bob a message, only Bob can read it (message confidentiality), Bob can 
verify that the message he received was in fact sent by Alice (message 
authenticity), and that it contains the data she sent (message integrity). 
Traditional email protocols provide none of these security properties 
out of the box. 

At the same time, users have come to expect certain features on top of 
traditional email.  These include automatic spam filtering, the ability 
to search and organize messages, and ubiquitous access to their email 
through webmail from a variety of user devices. We believe that providing 
stronger security guarantees cannot come at an expense of reduced usability. 
Our goal is to design an email service that provides the fundamental 
security properties of message confidentiality, integrity, and authenticity 
(called \emph{CIA guarantees} in the rest of the paper), while providing the 
features and user experience of webmail.  This is a nontrivial problem 
and current security approaches require running out-of-band security 
systems ``on top'' of email usually by leveraging public-key cryptography. 
These include S/MIME and PGP, as well as more recent ID-based encryption 
schemes~\cite{id-based-cryptography}.

The problem with out-of-band security approaches is that it significantly 
reduces usability by requiring active involvement in key management. 
Users must generate, distribute, and revoke public keys with the out-of-band 
system, and carefully guard their private keys while remaining vigilant 
for compromises. We believe this is unreasonable because most users do 
not understand practical information security. Even if they 
understood the security concerns of using webmail, using public-key cryptography in this 
manner greatly increases the complexity of basic email tasks. It's not surprising that a majority of users don't use security features even when they're available to them in the email client~\cite{garfinkel-email-survey}. 

Convenience of use is a critical design goal for any secure email system.
Our goal is to provide \emph{CIA guarantees} (confidentiality, integrity, and authenticity) without hurting usability.  
Our key insight is that email's store-and-forward approach makes \emph{CIA 
guarantees} hard to achieve. Because each email server stores and processes 
messages, a user must either trust the server completely or perform end-to-end authenticated encryption outside of the system. 
The former is unrealistic, but the latter requires users to set up and 
manage keys out-of-band. To address these challenges in this paper we 
present a new email system called StealthMail that enables secure email with usability semantics of webmail.

The key contributions of our work are 1) an automatic key management system 
(called AutoKey) and 2) a pull-based email exchange protocol (called Secure Message 
Request Protocol, or SMRP) that allow users to access their email using a 
web browser on any device. StealthMail uses AutoKey and SMRP to enable email access with only a 
username/password pair and a small amount of additional logic at the 
email client. To do so, StealthMail leverages the user's own personal cloud 
storage to host sealed account state, sealed keys, and sealed messages. 
The user devices (clients), not servers, process messages after decrypting 
them. Within StealthMail, the AutoKey system automatically distributes keys, 
e.g., Alice's public keys to Bob, using an array of non-colluding 
repositories. After key distribution, Bob can receive 
messages from Alice using SMRP. Under the hood, SMRP 
actually downloads messages from the sender's cloud storage. All the while, 
StealthMail remains transparently backwards-compatible with SMTP, and offers 
more limited security guarantees to non-StealthMail users.

The remainder of this paper is organized as follows.  In Section~\ref{sec:motivation}, 
we define our threat model and usability requirements, and argue why our strategy is necessary 
for meeting the usability requirements. We present the design of StealthMail, with a focus on AutoKey and SMRP,  
in Section~\ref{sec:design} and describe how basic 
email activities are performed. We present our implementation strategy for an 
open-source prototype (Section~\ref{sec:implementation}), and discuss results from a qualitative usability 
analysis and a preliminary evaluation (Section~\ref{sec:evaluation}).  
Finally, we discuss related work (Section~\ref{sec:related-work}) and future directions (Section~\ref{sec:conclusion}).
