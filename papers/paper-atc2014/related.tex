\section{Related Work}

Since inception of SMTP little had been done with respect to mail transport
protocol. Most of these works argue that design of SMTP incentivize SPAM as it
gives no control over receiver to what he gets in his inbox. It has also been
argued that in general push based architectures/protocols have a higher tendency
to attract unwanted traffic\cite{PushVsPull}. Therefore these works suggest a
pull based approach to email instead of the push based approach used by SMTP. 

DMTP (Differentiated Message Delivery Protocol) \cite{dtmp} is a protocol that
forces sender to store email in his own storage before recipient reads it into
his inbox unless the sender is known in advance. Only emails sent by senders
knowns to the recipient will be stored in recipient's mail box. Due to this
approach DMTP eliminates economies of scale and increases accountability of
SPAM.

In 2000 Internet Mail 2000 \cite{im2k} was proposed by Daniel J. Bernstein,
which shifts the responsibility of mail storage from recipient to sender through
a pull based approach that replaces SMTP which is analogous to conventional
smail mail. Original IM 2000 proposal does not specify mechanisms such as how
recipients should be notified, how messages should be downloaded etc. StubMail
\cite{stubmail} is an implementation of IM 2000, with StubMail sender uploads
the email to a permanently connected outbox (storage) and sends a UDP
notification to the recipient’s mail service provider as an RSS feed. Similar to
DMTP this implementation also shifts storage responsibility from recipient to
sender discouraging SPAMers. StubMail uses either HTTP \cite{HTTP-RFC} or 
SMTP\cite{SMTP-RFC} to upload mail to the outgoing storage, message notifications 
were sent over UDP to eliminate possibility of DoS attacks that could happen with 
TCP. HTMP (Hypertext Mail Protocol)\cite{htmp} too is based on IM 2000 and uses 
HTTP as a transport utilizing its’ PUT, GET and DELETE methods to handle email. 
In all of these designs email addresses of the sender and the receiver have to be 
exchanged out of band since receiver cannot receive emails if receiver does not 
know the location of sender’s mail OutBox.

However none of these designs were adopted as a main stream email protocol, this
could have been largely due to wide adoption of SMTP and availability of free
webmail services like GMail\cite{GMail}. At the time of inception there was no 
incentive in pushing storage responsibility to end - user when free webmail 
systems  were readily available with Gigabytes of storage. However during the 
last few years webmail users have become more conscious about privacy and fully 
aware of the fact that free storage comes at the cost of either being spied by 
the  government or becoming a target of an advertising campaign. On the other 
hand high availability online storages systems have become feasible enough for 
individual users through cloud storage services like Amazon S3\cite{AWS-S3},
Dropbox\cite{DROPBOX} etc. Therefore current technology and circumstances calls 
for a new mail transfer protocol that follows IM 2000 but accompanying novel 
mechanisms with respect to storage and privacy. In addition users are also keen 
on having full control over their content stored in the cloud storage and ability 
to move their data across different cloud storage providers. On the other hand by 
allowing users to plug-in their own cloud storage webmail services can liberate 
themselves from the burden of developing and maintaining high performance backend 
storage systems optimized for serving email.


